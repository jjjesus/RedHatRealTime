\documentclass[12pt]{article}

\usepackage{caption}
\usepackage{changepage} % adjustwidth
\usepackage{xcolor}
\usepackage{float}
\usepackage{graphicx}
\usepackage{fancyhdr}
\usepackage{listings} % Code listings
\usepackage{multicol}
\usepackage{multirow}

\usepackage[utf8]{inputenc}
\usepackage{setspace}

\usepackage[dvips,letterpaper,margin=1in]{geometry}

\setlength{\parindent}{15pt}
\setlength{\headheight}{15pt} % fancyhdr wants at least 14.5pt

% Header
\pagestyle{fancy}
\lhead{K3 Migration to Red Hat Real-Time Linux}
\rhead{August 28, 2020}
\renewcommand{\headrulewidth}{0.4pt}
\renewcommand{\footrulewidth}{0.4pt}

% Start document
\begin{document}

%%%%%%%%%%%%%%%%%%%% Title Page %%%%%%%%%%%%%%%%%%%%%%%%
\thispagestyle{empty}
\begin{titlepage}
\begin{center}
        \vspace*{1cm}

        \LARGE{Ku Band Radio Frequency System Version 3 (K3) \\
        Migration from Concurrent RedHawk Linux to \\
        Red Hat Real-Time Linux}

        \vspace{0.5cm}
        \LARGE
        % Subtitle

        \vspace{1.5cm}

        \normalsize

        John Jesus \\
        August 28, 2020

        \vfill



        \vspace{0.8cm}




\end{center}
\end{titlepage}

\tableofcontents
\newpage

%%%%%%%%%%%%%%%%% Start of Body %%%%%%%%%%%%%%%%%%%%%%%%
\section{Scope}
\subsection{Identification}
This document is a Raytheon internal document analyzing an opportunity to change the version of Linux operating system used in the Ku Band Radio Frequency System Version 3 (K3).

\subsection{Linux operating system overview}

\subsection{Document purpose}

\subsection{Document overview}
The structure of this document:

\begin{enumerate}
    \item Document scope (this section)
    \item References
    \item Interfaces
    \item Analysis of impact of severing the interfaces between the SCU
    SBC and the SCIM in K3
\end{enumerate}



\section{References}

\begin{enumerate}
    \item XPedite7674 Users Manual Revision C (Extreme Engineering Solutions, February 1, 2018) \label{ref:board_man}
    \item XPedite7674 Schematic Diagram SCH90030490 Revision C (Extreme Engineering Solutions, February 28, 2020) \label{ref:7674schematic}
    \item System Controller Unit Interface Module - Schematic Diagram SDD0862497 (Raytheon, May 14, 2020) \label{ref:scim_schematic}
    \item Development Specification for the Ku Band RF System (KRFS) System Controller Unit (SCU) B4597324 (Raytheon, March 30, 2017) \label{ref:dev_scu}
    \item Development Specification for the Ku Band RF System (KRFS) System Controller Interface Module (SCIM) B4805956 (Raytheon, April 11, 2017) \label{ref:dev_scim}
    \item Ku-Band RF System (KRFS) Power Control Software Thin Specification (Raytheon, April 7, 2017) \label{ref:power_control_sw}
    \item Ku-Band RF System (KRFS) System Controller Interface Module (SCIM) FPGA Design Document (FDD) U1011477 (Raytheon, February 6, 2017) \label{ref:scim_fdd}
\end{enumerate}


\section{Interfaces between the SCU-SBC and the SCIM}
\label{sec:interfaces}

The interfaces between the SCU-SBC and the SCIM are depicted in Figure \ref{fig:interface}.
Applications on the SBC can access one of the two drivers that each expose an interface between
the SBC and the SCIM:

\begin{enumerate}
    \item Serial port interface through device \texttt{/dev/ttyS2}
    \item PCIe interface through device \texttt{/dev/drvScimFpgaChar0}
\end{enumerate}


Each of these interfaces are custom Linux kernel modules (device drivers) software from
Raytheon.
The serial port interface is based on exchanging a data structure over file \emph{reads/writes}.
The PCIe interface is based on exchanging a different data structure from that of the serial port using the file \emph{ioctl} operation.

\subsection{Serial port interface}

To use the serial port interface to the SCIM, software invokes file operations to open \texttt{/dev/ttyS2} and write data to it. The structure
of the data is in Listing \ref{lst:serial_msg}.

% Default formatting for C++ listing
\lstset
{
    language=C++,
    basicstyle=\footnotesize\ttfamily,
    linewidth=7in,
    %basicstyle=\small\ttfamily,
    numbers=left,
    stepnumber=1,
    showstringspaces=false,
    tabsize=1,
    breaklines=true,
    breakatwhitespace=false,
}

\begin{adjustwidth}{0.0in}{0.0in}
\begin{lstlisting}[firstnumber=143, caption={\texttt{exeCswBoot/SerialCommonTypes.h} Serial Port message (variable length)}, label={lst:serial_msg}]
//********************************************************************//
//           HEADER STRUCTURE FOR SENDING/RECV MSG                    //
//********************************************************************//
#pragma pack(1)
typedef struct
{
   UInt8     headerChecksum;    //Package header checksum
   UInt8     dataChecksum;      //Package data checksum
   UInt16    sf2Cmd;            //Type of the command or response
   UInt8     isACommand;        //Use on ensure if this is a response of command package
   UInt8     Src_LRU;           //Source
   UInt8     Src_SLOT;          //Source
   UInt8     Dst_LRU;           //Used in command response  to match
   UInt8     Dst_SLOT;          //Used in command response to match
   UInt16    msgSize;           //The whole message size (header + data payload)
   UInt8     cmdIndex;          //Increasing every command of the same type. Used in command response to match
} SF2PacketHeader_Type;
#pragma pack(8)

//********************************************************************//
//             RECEIVING MESSAGE STRUCTURE WITH HEADER                //
//********************************************************************//
#pragma pack(1)
typedef struct
{
   SF2PacketHeader_Type    sf2Header;
   void*                   sf2Payload;
} SF2DataMsg_type;
#pragma pack(8)

\end{lstlisting}
\end{adjustwidth}


The message sent over the serial port has a command word (Line 151 of Listing \ref{lst:serial_msg}) to either query data (get version,
write-protect status, or system mode) or to send a signal to the rest of the radar (set maintenance mode, set write-protect, shutdown radar).

\begin{table}[H]
    \captionsetup{width=.9\linewidth}
    \caption{Serial port commands from exeCswBoot/SerialCommonTypes.h}
    \resizebox{\textwidth}{!}{%
    \begin{tabular}{p{4.0in}p{3.0in}}
    \hline
        Command & Description/Purpose \\
    \hline
        SF2CMD\_DISCOVER&
        \multirow{3}{=}{Query for the tuple containing (my LRU id, my Slot id, write-protect status, system mode)}\\
        SF2CMD\_DISCOVER\_RESP\\
        \\
    \hline
        SF2CMD\_GET\_OPERATION\_MODE&
        \multirow{4}{=}{Get/Set System Mode (Tactical/Maintenance)}\\
        SF2CMD\_GET\_OPERATION\_MODE\_RESP\\
        SF2CMD\_SET\_OPERATION\_MODE\\
        SF2CMD\_SET\_OPERATION\_MODE\_RESP\\
    \hline
        SF2CMD\_GET\_WRITE\_PROTECT&
        \multirow{4}{=}{Get/Set Write-protect Status}\\
        SF2CMD\_GET\_WRITE\_PROTECT\_RESP\\
        SF2CMD\_SET\_WRITE\_PROTECT\\
        SF2CMD\_SET\_WRITE\_PROTECT\_RESP\\
    \hline
        SF2CMD\_SHUTDOWN&
        \multirow{2}{=}{Shutdown radar}\\
        SF2CMD\_SHUTDOWN\_RESP\\
    \hline
        SF2CMD\_GET\_VERSION&
        \multirow{6}{=}{Miscellaneous status (version, flash CRC, uptime)}\\
        SF2CMD\_GET\_VERSION\_RESP\\
        SF2CMD\_FLASH\_CRC\\
        SF2CMD\_FLASH\_CRC\_RESP\\
        SF2CMD\_UPTIME\\
        SF2CMD\_UPTIME\_RESP\\
    \hline
        SF2CMD\_PROGRAM&
        \multirow{6}{=}{Unused at SCU (the ABEU uses these to download or re-program its FPGA flash)}\\
        SF2CMD\_PROGRAM\_RESP\\
        SF2CMD\_VERIFY\\
        SF2CMD\_VERIFY\_RESP\\
        SF2CMD\_SET\_WRITE\_PROTECT\_FACTORY\\
        SF2CMD\_SET\_OPERATION\_MODE\_FACTORY\\
        SF2CMD\_UPLOAD\\
        SF2CMD\_UPLOAD\_RESP\\
        SF2CMD\_DOWNLOAD\\
        SF2CMD\_DOWNLOAD\_RESP\\

    \hline
    \end{tabular}%
    }
    \label{tab:serial_cmd}
\end{table}

\subsubsection{Applications using the serial port interface}
On the SCU-SBC, only one or two command-line executables use the serial port interfaces, and, these are only used in a terminal or ssh session as troubleshooting or debug commands. The serial port interface seems to be used more on the ABEU rather than on the SCU; software on the SCU seems to prefer using its exclusive PCIe interface instead of the serial port.

The serial port commands depicted as \emph{SF2 App} near the upper-right of Figure \ref{fig:interface} are:

\begin{enumerate}
    \item \texttt{sf2shutdown}
    \item \texttt{sf2network}
\end{enumerate}


\subsection{PCIe interface}

See Reference \ref{ref:scim_fdd}.

\begin{table}[H]
    \captionsetup{width=.9\linewidth}
    \caption{SCIM registers available over PCIe from drvScimFpga/drvScimIoctl.h}
    \resizebox{\textwidth}{!}{%
    \begin{tabular}{p{2.0in}p{5.0in}}
    \hline
        \textbf{Register} & \textbf{Description/Purpose} \\
    \hline
        Test Register & Scratch register, no operational role \\[1.0mm]
        Firmware version & SCIM firmware version string \\[1.0mm]
        Power discretes & Bitmap with each bit indicating the status of
                          a power rail (see Table \ref{tab:power_discretes})\\[1.0mm]
        System discretes & Bitmap used to manage and detect cooling flow to each APSU, ABEU, Array, and the SCU \\[1.0mm]
        CCA discretes & Bitmap used to manage the LEDs on the SCIM \\[1.0mm]
        Temperature, voltage, and humidity & Measurement values from various
                            environmental sensors on the SCIM, including high-low-snapshot values on each power rail, high-low-snapshot temperature values, humidity, and accelerometer.  \\[1.0mm]
        Zeroize & Bitmap to zeroize the SCU solid-state drive, GPS, and Ethernet switch.\\[1.0mm]
        Power status & Bitmap to monitor the lift and to indicate good/fail of 3.3V, 5V, 8V, 12V, Core Power Supply and each APSU \\[1.0mm]
        LPDDR status & Status of the Low-Power Dual Data Rate (LPDDR) memory on the SCIM, include memory test command and results \\[1.0mm]
        Software CSR & Control and status register of SCIM software.  Bitmap that communicates important conditions (see Table \ref{tab:scim_csr}) \\[1.0mm]
        Application Software CSR & Control and status register (CSR) of application software.  Bit 0 indicates that the application software has compelted a graceful shutdown.\\[1.0mm]
        Platform ID, Version information and uptime & Five registers contain Platform ID (K2, K3, XBAEU), SCIM software version, SCU serial number, SCIM serial number, and SCIM elapsed uptime \\[1.0mm]
        PMBus control and status & Power Management Bus control and status registers \\[1.0mm]
        PDCU control and status & The SCIM uses these registers to communicate on a serial port with the Power Distribution Control Unit (PDCU) \\
    \hline
    \end{tabular}%
    }
    \label{tab:pcie_regs}
\end{table}

\begin{table}[H]
    \captionsetup{width=.9\linewidth}
    \caption{Power discretes}
    %\resizebox{\textwidth}{!}{%
    \begin{tabular}{p{0.5in}p{5.5in}}
    \hline
        Bit & Description \\
    \hline
        8& Set bit to enable the Beacon Power Supply.\\
        7& Clear bit to reset and configure the GPS.\\
        6& Clear bit to disable the GPS Auxiliary Power Supply.\\
        5& Clear bit to disable the GPS Primary Power Supply.\\
        4& Clear bit to enable the GPS Primary Power Supply.\\
        3-0& Set each bit to enable the corresponding ABEU (3:0) Power Supply.\\
    \hline
    \end{tabular}%
    %}
    \label{tab:power_discretes}
\end{table}

\begin{table}[H]
    \captionsetup{width=.9\linewidth}
    \caption{Power discretes}
    %\resizebox{\textwidth}{!}{%
    \begin{tabular}{p{0.5in}p{5.5in}}
    \hline
        Bit & Description \\
    \hline
        5 & Over-temperature fault exists.  The SCIM will cut power in two minutes if ABEUs are powered on or immediately cuts power if ABEUs are powered off. \\
        4 & PMBus interface test has failed. The 28V voltage value read from the power supply is out of range. \\
        3 & PMBus interface test has passed.  The 28V voltage value read from the power supply is OK. \\
        2 &  Hard drive is inserted into SCU. The NVMRO (write-protect) cannot be enabled if this bit indicates that the hard drive is still inserted in the SCU. \\
        1 & SCIM flash memory CRC check failed. \\
        0 & SCIM flash memory CRC check passed. \\
    \hline
    \end{tabular}%
    %}
    \label{tab:scim_csr}
\end{table}

\subsubsection{Applications using the PCIe interface}
Any software application can open the device \texttt{/dev/drvScimFpgaChar0} and invoke \emph{ioctl} on it to interface with the SCIM.  Only one application on the SCU-SBC actually does so, the \emph{PowerControl} application.  This application originally used the interface to provide a safety-interlock feature for power (checks that coolant is flowing before powering up an ABEU), but, has morphed into a kind of server for access to any/all of the registers and features of the SCIM (given this expanded role, the application seems to have outgrown its original \emph{PowerControl} name).


The \emph{PowerControl} application fulfills requests received as Common Operating Environment (COE) messages from external applications
(see Table \ref{tab:coe_msgs}).  In a sense, these COE messages have become the \emph{real} API to the SCIM.  This API is especially
important in powering up the radar system as discussed in Section \ref{sec:power-up}.

\begin{table}[H]
    \captionsetup{width=.9\linewidth}
    \caption{COE Messages for the PowerControl software}
    \resizebox{\textwidth}{!}{%
    \begin{tabular}{p{2.1in}p{1.2in}p{1.2in}p{2.0in}}
    \hline
        Message & Sender & Receiver & Description \\
    \hline
PCU\_KRFS\_Command & TRUI & Power Control & This command powers on the radar system.  \\
PCU\_Status & Power Control & TRUI, MFRFS\_Health & This message is sent every 1 second to send coolant flow status to the TRUI.  \\
    \hline
PCU\_ClearStatusReq & TRUI & Power Control & The TRUI is acknowledging the fault latches and requesting to clear the latches.\\
PCU\_ClearStatusResp & Power Control & TRUI & PowerControl returns the value of the fault latches. \\
    \hline
PCU\_GPS\_PowerReq & TRUI & Power Control & The TRUI is requesting a power change on the GPS.  Some safety interlock logic is applied within the Power Control application. \\
PCU\_GPS\_PowerResp & Power Control & TRUI & Status result from the GPS power request. \\
    \hline
PCU\_SCU\_PowerSupplyCmd & MFRFS\_Health & Power Control & This message is used by BIT to request a shutdown of the radar. \\
    \hline
PCU\_ControlCommand & TRUI & Power Control & This command is used to request zeroize of the GPS. \\
PCU\_ControlStatus & Power Control & TRUI & Response to the zeroize command is to send back power status. \\

    \hline
    \end{tabular}%
    }
    \label{tab:coe_msgs}
\end{table}

\subsubsection{Use Case: Powering on the KRFS radar}
\label{sec:power-up}.

Powering on the KRFS radar is depicted in the sequence diagram of Figure \ref{fig:power-up}.  As indicated in the figure, the Operator throws circuit breakers
that power on the cooling system and the SCU (which includes the SBC, SCIM, Hard Drive, Ethernet Switch).  The Operator also connects the Tactical Radar User Interface (TRUI) laptop to the Ethernet switch to establish connection between the TRUI and the SCU SBC.

The sequence then proceeds as described in the diagram, with the PowerControl application on the SBC exchanging COE messages with the laptop and accessing the SCIM over the PCIe interface.

The bottom of Figure \ref{fig:power-up} depicts the periodic (1 Hz) polling
by the PowerControl application of various discrete status on the SCIM that
are broadcast to the TRUI over COE; these messages are also consumed by a
system health/BIT application running on the SCU SBC (not shown).

Eventually, logic in PowerControl determines from status of the discretes (from Figure \ref{fig:power-up} Message 14) that the system is ready to power on the arrays and exchanges messages with the SCIM to effect power on of the ABEUs.


\section{Analysis of required interface changes}


\subsection{What must change: SCU-SBC Serial Port and PCIe interfaces to the SCIM interfaces are eliminated}

For K3, the existing SCU-SBC PCIe and Serial Port interfaces to the SCIM are eliminated.

There was some initial thinking that we might be able to operate with reduced functionality (for example, with degraded power/health reporting), but, the content of Section \ref{sec:interfaces} indicates that we need the SBC-SCIM interface even to turn the system on, among other things.

\subsection{Proposed solution: Replace SCIM PCIe interface with Ethernet}
The proposed solution is to replace the SBC-SCIM PCIe interface with Ethernet.
This solution would not require hardware change on the SBC which already supports several Ethernet connections.
It does require a change to the SCIM SOCF and IO Flex to route Ethernet instead of PCIe.  The SCIM SOCF (Microsemi SmartFusion 2 (SF2) System-on-a-Chip FPGA) is depicted in \ref{fig:socf-eth}; this figure from the vendor indicates available modules for triple-speed Ethernet (TSE) over SERDES (without magnetics).



\end{document}

