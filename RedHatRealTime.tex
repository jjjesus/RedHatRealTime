\documentclass[12pt]{article}

\usepackage{caption}
\usepackage{changepage} % adjustwidth
\usepackage{xcolor}
\usepackage{float}
\usepackage{graphicx}
\usepackage{fancyhdr}
\usepackage{listings} % Code listings
\usepackage{multicol}
\usepackage{multirow}

\usepackage[utf8]{inputenc}
\usepackage{setspace}

\usepackage[dvips,letterpaper,margin=1in]{geometry}

\setlength{\parindent}{15pt}
\setlength{\headheight}{15pt} % fancyhdr wants at least 14.5pt

% Header
\pagestyle{fancy}
\lhead{K3 Migration to Red Hat Real-Time Linux}
\rhead{August 28, 2020}
\renewcommand{\headrulewidth}{0.4pt}
\renewcommand{\footrulewidth}{0.4pt}

% Start document
\begin{document}

%%%%%%%%%%%%%%%%%%%% Title Page %%%%%%%%%%%%%%%%%%%%%%%%
\thispagestyle{empty}
\begin{titlepage}
\begin{center}
        \vspace*{1cm}

        \LARGE{Ku Band Radio Frequency System Version 3 (K3) \\
        Migration from Concurrent RedHawk Linux to \\
        Red Hat Real-Time Linux}

        \vspace{0.5cm}
        \LARGE
        % Subtitle

        \vspace{1.5cm}

        \normalsize

        John Jesus \\
        August 28, 2020

        \vfill



        \vspace{0.8cm}




\end{center}
\end{titlepage}

\tableofcontents
\newpage

%%%%%%%%%%%%%%%%% Start of Body %%%%%%%%%%%%%%%%%%%%%%%%
\section{Scope}
\subsection{Identification}
This document is a Raytheon internal document analyzing an opportunity to change the version of Linux operating system used in the Ku Band Radio Frequency System Version 3 (K3).

\subsection{Linux operating system overview}

\subsection{Document purpose}

\subsection{Document overview}
The structure of this document:

\begin{enumerate}
    \item Document scope (this section)
    \item References
    \item RedHawk features used in current K2 software
    \item Migrating to Red Hat Real-Time in K3
\end{enumerate}



\section{References}

\begin{enumerate}
    \item RedHawk Linux Users Guide 0898004-780 (Concurrent Computer Corporation, March 2016) \label{ref:red_hawk_guide}
    \item RedHawk Architect Users Guide  0898601-7.2-1 (Concurrent Computer Corporation, December 2016) \label{ref:architect}
\end{enumerate}

\section{RedHawk features used in current K2 software}
\label{sec:redhawk_features}

\section{Migrating to Red Hat Real-Time in K3}
\label{sec:redhat_migration}


\end{document}

