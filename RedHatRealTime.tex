\documentclass[12pt]{article}

\usepackage{array} % table right-justify
\usepackage{caption}
\usepackage{changepage} % adjustwidth
\usepackage{colortbl} %
\usepackage{enumitem} %
\usepackage{xcolor}
\usepackage{float}
\usepackage{graphicx}
\usepackage{fancyhdr}
\usepackage{listings} % Code listings
\usepackage{multicol}
\usepackage{multirow}
\usepackage{xurl}
\usepackage{setspace}
\usepackage{etoolbox}

\AtBeginEnvironment{quote}{\singlespacing\small}
%\AtBeginEnvironment{quote}{\small}

\usepackage[utf8]{inputenc}
\usepackage{setspace}

\usepackage[dvips,letterpaper,margin=1in]{geometry}

\setlength{\parindent}{15pt}
\setlength{\headheight}{15pt} % fancyhdr wants at least 14.5pt
\setlist{noitemsep}

\definecolor{light-gray}{gray}{0.95}

% Header
\pagestyle{fancy}
\lhead{K3 Migration to Red Hat Real-Time Linux}
\rhead{August 28, 2020}
\renewcommand{\headrulewidth}{0.4pt}
\renewcommand{\footrulewidth}{0.4pt}

% Start document
\begin{document}

%%%%%%%%%%%%%%%%%%%% Title Page %%%%%%%%%%%%%%%%%%%%%%%%
\thispagestyle{empty}
\begin{titlepage}
\begin{center}
        \vspace*{1cm}

        \LARGE{Ku Band Radio Frequency System Version 3 (K3) \\
        Migration from Concurrent RedHawk Linux to \\
        Red Hat Real-Time Linux}

        \vspace{0.5cm}
        \LARGE
        % Subtitle

        \vspace{1.5cm}

        \normalsize

        John Jesus \\
        August 28, 2020

        \vfill



        \vspace{0.8cm}




\end{center}
\end{titlepage}

\tableofcontents

%%%%%%%%%%%%%%%%% Scope %%%%%%%%%%%%%%%%%%%%%%%%
%
\newpage
\section{Scope}
\subsection{Identification}
This document is a Raytheon internal document analyzing an opportunity to change the version of
Linux operating system used in the Ku Band Radio Frequency System (KRFS) Version 3 (K3).

\subsection{System overview}
KRFS is a radar system that supports the counter- rocket, artillery, and mortar (CRAM) mission.
KRFS has four radar apertures each operating as an Active Electronically
Scanned Array (AESA).  Each aperture has its own receiver/exciter and signal processing
subsystem called the Array Backend Electronics Unit (ABEU).  The arrays are arranged to
provide 360 degree coverage in azimuth and horizon to 90-degree coverage in elevation.

\begin{figure}[H]
    \begin{center}
    \includegraphics[width=1.0\textwidth]{img/k3}
    \caption{K3 Radar System}
    \label{fig:k3}
    \end{center}
\end{figure}

The System Controller Unit (SCU) is a 3-U VPX/VME shelf with commercial-off-the-self (COTS)
cards and one custom card assembly (CCA).  The SCU is the central controller for the radar system,
and the main control software within the SCU runs in its SBC card (Figure \ref{fig:scu}).

\begin{figure}[H]
    \begin{center}
    \includegraphics[width=1.0\textwidth]{img/scu}
    \caption{The SBC card in the SCU}
    \label{fig:scu}
    \end{center}
\end{figure}

In KRFS K2, the SBC Processor is a Extreme Engineering (XES) Xpedite7672 card based on the Xeon-D
System-on-a-Chip (SOC) running the Linux operating system.
In KRFS K3, the SBC Processor is a very similar Extreme Engineering (XES) Xpedite7674 card also based on the Xeon-D SOC
and running the Linux operating system.

\subsection{Overview of Linux operating system and real-time features}
The Linux operating system is a free, open-source software (FOSS) operating system originally authored by Linus Torvalds in September 1991 as a Unix system for his personal computer.  Linux eventually became the operating system underlying platforms from the smallest devices
(Android phones are based on Linux) to supercomputers. Although the source code for the software is freely available,
some commercial vendors charge fees to provide the software pre-compiled for a particular device, board, or machine,
and charge for support and additional documentation or other tools.
Linux initially resembled the Unix systems of its time; it supported multiple users and multiple simultaneously running programs and used a \emph{completely fair scheduler} (CFS) to insure equality among the running processes.

\subsubsection{Real-time patch (RT Patch) for the Linux operating system}
In 2005, a patch was released for the Linux source code to support real-time processing.
In a real-time system like a radar, some processing has deadlines, for example, transmitting pulses or processing signal returns, where instead of emphasizing fair scheduling among all processes, a real-time operaing system (RTOS) must assign priorities to tasks and interrupt (preempt) not only the currently running task but even its own internal mechanisms (spinlocks, semaphores, and interrupt handling) to meet the real-time deadlines.
The original patch was originally named \emph{PREMPT\_RT} or \emph{CONFIG\_PREEMPT\_RT} and authored by some of the most important Linux kernel developers, Ingo Molnar and Thomas Gleixner of Red Hat and Paul McKenny of IBM.
The source code for the RT Patch has been actively maintained separately from the mainline Linux source code over the last 15 years by teams led by these same developers and has been downloaded and incorporated in thousands of systems and other Linux software distributions.

\subsubsection{SuSE Linux Enterprise Real Time}
Also around 2005, the German firm \emph{Software und System-Entwicklung} (SuSE) released \emph{SuSE Linux Enterprise Real Time} which built on the RT Patch and added some features:

\begin{enumerate}
    \item CPU shielding - in multiprocessing systems, CPUs can be shielded from interrupt processing and operaing system background tasks so that they can be dedicated to real-time processing
    \item Support for a Real-Time Clock and Interrupt Module (RCIM) - a PCI card that provided programmable timers and 12 input and output external interrupt lines to synchronize clocks across systems.  SuSE provided Linux device drivers for this card.
    \item A Frequency Based Scheduler (FBS) - SuSE provided a scheduler based on the high-resolution timer provided by the RCIM to support software that performed high-frequency short-deadline processing.  SuSE also released a debugging tool providing a view of system utilization during each frequency slice of the scheduler.
\end{enumerate}
\label{ref:suse_rt_features}


\subsubsection{Concurrent Computing Corporation RedHawk Linux}


This comment from the Linux Weekly News online is representative of
many in the Linux community to Concurrent's announcement of its real-time
product:

\begin{quote}
Parent article: Concurrent releases RedHawk Linux 2.1
They do not have to release source to the world, but,
they certainly do have the obligation to make source
available to those to whom they're distributing these
customized Linux kernels. The company seems to be very quiet on these obligations,
to judge from their web site. I think it's entirely
appropriate that they get some scrutiny from the folks
whose code they've customized.\footnote{\url{https://lwn.net/Articles/82589/}}
\end{quote}


\subsubsection{RedHat Linux for Real-Time (Red Hat RT)}


\subsubsection{Merge of RT Patch to Linux mainline in July 2019}

\subsection{Document purpose}
To integrate NE software, we need to be able to build Linux from source.  NE currently uses Red Hat Real-Time.
 We use Red Hawk from Concurrent RTI.  NE builds from source using tools from Red Hat.  We do not build from source.

Produce a document that helps us decide how to get our Linux image created.


\subsection{Document overview}
The structure of this document:

\begin{enumerate}
    \item Document scope (this section)
    \item References
    \item Real-time features from RedHawk used in current KRFS software
    \item Creating the Linux operating system image
    \item Hardening the Linux operating system image
    \item Migrating to Red Hat Real-Time in K3
\end{enumerate}



%%%%%%%%%%%%%%%%% References %%%%%%%%%%%%%%%%%%%%%%%%
%
\newpage
\section{References}

\begin{enumerate}
    \item \textit{RedHawk Linux Users Guide 0898004-780} (Concurrent Computer Corporation, March 2016) \label{ref:red_hawk_guide}
    \item \textit{RedHawk Architect Users Guide 0898601-7.2-1} (Concurrent Computer Corporation, December 2016) \label{ref:architect}
    \item \textit{NightStar RT Tutorial Version 4.4 0898009-090} (Concurrent Computer Corporation, May 2014) \label{ref:nightstar}
    \item \textit{Red Hat Enterprise Linux for Real Time 7 Reference Guide} (Red Hat Corporation, November 6, 2015) \label{ref:rhel7_ref}
    \item \textit{Red Hat Enterprise Linux for Real Time 7 Installation Guide} (Red Hat Corporation, November 6, 2015) \label{ref:rhel7_install}
    \item Source for Red Hawk costs in Table \ref{tab:license_costs} was Karl Weis, Global Supply Chain Program Lead, Raytheon \label{ref:karl}
    \item Source for Red Hat Real-Time costs in Table \ref{tab:license_costs} was Lynn Bonenfant, IT Program Lead, Raytheon \label{ref:lynn}
    \item \textit{Pull CONFIG\_PREEMPT\_RT stub config from Thomas Gleixner} commit message from the web view of the Linux git repository at \url{https://git.kernel.org/pub/scm/linux/kernel/git/torvalds/linux.git/commit/?id=70e6e1b971e46f5c1c2d72217ba62401a2edc22b} \label{ref:commit}
\end{enumerate}


%%%%%%%%%%%%%%%%% RedHawk Features %%%%%%%%%%%%%%%%%%%%%%%%
%
\newpage
\section{Real-time features from RedHawk in KRFS}
\label{sec:redhawk_features}

\begin{table}[H]
    \captionsetup{width=.9\linewidth}
    \caption{RedHawk real-time feature list (Part 1)}
    \resizebox{\textwidth}{!}{%
    \begin{tabular}{p{2.8in}p{1.2in}p{3.6in}}
    \hline
        \textbf{Feature} & \textbf{Used in KRFS} & \textbf{Description} \\
    \hline
        \rowcolor{light-gray} Processor Shielding & \textbf{Yes} & The term and mechanisms for \emph{shielded CPU} comes from the SuSE Linux Enterprise Real Time release of April 2008 for constaining the running of operating system threads on certain CPUs in a multicore/multiprocessor system. KRFS uses this feature to protect or \emph{shield} certain CPUs from running Linux kernel lower-priority operations and lower-priority interrupts (such as disk I/O) so that the shielded CPUs can be dedicated to real-time application (radar processing) code.  The standard Linux kernel equivalent is named \emph{cpuset} and works in a similar way.\\
        \rowcolor{light-gray} Processor Affinity & \textbf{Yes} & This feature enables application workload to be spread among multiple CPUs to exploit concurrent processing.  This assignment of application processes to \emph{cpusets} is specified in userspace (that is, the application programmer decides this partitioning).  In KRFS, signal processing application threads can be distributed among shielded CPUs so dedicated and network communications can be partitioned onto other cores/CPUs. \\[4.5mm]
User-level Preemption Control & No & User control of preemption of userspace synchronization mechanisms like POSIX semaphores using a system function call \emph{resched\_cntl}.  Makes kernel space preemption mechanism invocable from userspace. \\
Fast Block/Wake Services & No & A preemptible semaphore in the kernel available from RT Patch since 2007. \\
RCIM Driver & No & Concurrent sells a PCI card called the Real-time Clock Interface Module (RCIM).  The card can provide a 400ns clock signal to the operating system through an interrupt handler to enable a high-resolution timer feature. \\
Frequency-Based Scheduler (FBS) & No & Source code for the FBS provided by Concurrent is actually copied without attribution from the original SuSE Linux Real-Time Extensions released in 2006. \\
/proc Modifications & No & This feature is part of the \emph{usermap} feature (see below) that can be used by Nightstar or even regular users.\\
Kernel Trace Facility & No & This feature is combined with the standard kernel debuggers to support enhanced kernel debugging features using Nightstar \\
ptrace Extensions & No & This feature is combined with the \emph{usermap} feature (below) to support Nightstar debugging) \\
    \hline
    \end{tabular}%
    }
    \label{tab:redhawk_features_1}
\end{table}


\begin{table}[H]
    \captionsetup{width=.9\linewidth}
    \caption{RedHawk real-time feature list (Part 2)}
    \resizebox{\textwidth}{!}{%
    \begin{tabular}{p{2.8in}p{1.2in}p{3.6in}}
    \hline
        \textbf{Feature} & \textbf{Used in KRFS} & \textbf{Description} \\
    \hline
Kernel Preemption & (Yes,built-in) & The RT Patch is listed as a Concurrent RedHawk feature though it obviously is sourced from the RT Patch (and now the Linux mainline, see Appendix A). \\[4.5mm]
Real-Time Scheduler & (Yes,built-in) & Alternatives to the Linux default Completely Fair Scheduler (CFS) like priority-based SCHED\_FIFO and SCHED\_RR (round-robin) have been part of the RT Patch since 2007.  SCHED\_DEADLINE is available in Linux 3.6 (Red Hat 7.6) and later. \\[4.5mm]
Low Latency Enhancements & (Yes,built-in) & Some kernel tuning derived from Red Hat's TUNA and SuSE Linux tools have been added as default. \\[4.5mm]
Priority Inheritance & (Yes,built-in) & This has been part of SuSE Linux and the RT Patch since 2007.  \\[4.5mm]
High Resolution Process Accounting & No & This feature follows from the use of the RCIM card and driver (above).\\[4.5mm]
Capabilities Support & No & Permission files (\emph{capabilities} files) for features like the \emph{usermap} and \emph{/proc mmap} features (below). \\[4.5mm]
Kernel Debuggers & (Yes,built-in) & The built-in kernel debuggers \emph{kdb} and \emph{kgdb} can be configured to work with Nightstar. \\[4.5mm]
Kernel Core Dumps/Crash and Live Analysis & (Yes,built-in) &  This feature supports Nightstar, allowing it to directly map and introspect kernel memory. \\[4.5mm]
User-level Spin Locks & No &  User control of preemption of spinlocks in the kernel provided by a special system function call \emph{resched\_cntl}.  Makes kernel space preemption mechanism invocable from userspace. \\[4.5mm]
usermap and /proc mmap & No & This feature supports Nightstar, allowing it to directly map and introspect any user process.  \\[4.5mm]
Hyper-threading & (Yes,built-in) & This is a setting in the CPU hardware that is controlled in the BIOS or bootloader that allows in-core multi-processing.  Support in mainline Linux for hyper-threading and symmetric multi-processing (SMP) has been since 2004.  \\[4.5mm]
XFS Journaling File System & (Yes,built-in) & %
The XFS filesystem was invented at Silicon Graphics (SGI) in 1996 and has been available for the Linux kernel since 2001.%
\\
    \hline
    \end{tabular}%
    }
    \label{tab:redhawk_features_2}
\end{table}

\subsection{Equivalent real-time features from RedHat RT for K3}

%%%%%%%%%%%%%%%%% Image creation %%%%%%%%%%%%%%%%%%%%%%%%
%
\newpage
\section{Creating the Linux operating system image}
\label{sec:image_creation}

\subsection{Creating the Linux operating system image in K3}


%%%%%%%%%%%%%%%%% Hardening %%%%%%%%%%%%%%%%%%%%%%%%
%
\newpage
\section{Hardening the Linux operating system image}
\label{sec:image_hardening}

\subsection{Hardening the Linux operating system image in K3}


%%%%%%%%%%%%%%%%% Migrating to Red Hat %%%%%%%%%%%%%%%%%%%%%%%%
%
\newpage
\section{Migrating to Red Hat Real-Time in K3}
\label{sec:redhat_migration}


\subsection{Build artifact changes}


\subsection{Source code changes}


\subsection{License costs}

Table \ref{tab:license_costs} summarizes license costs (see References \ref{ref:karl} and \ref{ref:lynn}).

\begin{table}[H]
\captionsetup{width=.9\linewidth}
\caption{License costs}
\resizebox{\textwidth}{!}{%
\begin{tabular}{%
    >{\raggedright\arraybackslash}p{3.2in}%
    >{\raggedleft\arraybackslash}p{1.0in}%
    >{\raggedleft\arraybackslash}p{1.0in}}
\hline
                                 & Red Hawk & Red Hat RT \\
\hline
    Product Maintenance (yearly) & 281,000 & 0 \\
    Development License (yearly) &  20,796 & 2,700 \\
\hline
    Run-time License (perpetual, per-instance) & 300 & 0 \\
    Run-time License (yearly, per-instance) & 400 & 2,700 \\
\hline
\end{tabular}%
}
\label{tab:license_costs}
\end{table}


%%%%%%%%%%%%%%%%% Appendix A: Merge to mainline %%%%%%%%%%%%%%%%%%%%%%%%
%
\newpage
\section{Appendix A: July 2019 Merge of RT to Linux mainline}

The git commit message from Linus Torvalds merging the real-time patch
into the Linux mainline on July 20, 2019 is below (Reference \ref{ref:commit}).

\begin{verbatim}
Pull CONFIG_PREEMPT_RT stub config from Thomas Gleixner:
 "The real-time preemption patch set exists for almost 15 years now and
  while the vast majority of infrastructure and enhancements have found
  their way into the mainline kernel, the final integration of RT is
  still missing.

  Over the course of the last few years, we have worked on reducing the
  intrusivenness of the RT patches by refactoring kernel infrastructure
  to be more real-time friendly. Almost all of these changes were
  benefitial to the mainline kernel on their own, so there was no
  objection to integrate them.

  Though except for the still ongoing printk refactoring, the remaining
  changes which are required to make RT a first class mainline citizen
  are not longer arguable as immediately beneficial for the mainline
  kernel. Most of them are either reordering code flows or adding RT
  specific functionality.

  But this now has hit a wall and turned into a classic hen and egg
  problem:

     Maintainers are rightfully wary vs. these changes as they make only
     sense if the final integration of RT into the mainline kernel takes
     place.

  Adding CONFIG_PREEMPT_RT aims to solve this as a clear sign that RT
  will be fully integrated into the mainline kernel. The final
  integration of the missing bits and pieces will be of course done with
  the same careful approach as we have used in the past.

  While I'm aware that you are not entirely enthusiastic about that, I
  think that RT should receive the same treatment as any other widely
  used out of tree functionality, which we have accepted into mainline
  over the years.

  RT has become the de-facto standard real-time enhancement and is
  shipped by enterprise, embedded and community distros. It's in use
  throughout a wide range of industries: telecommunications, industrial
  automation, professional audio, medical devices, data acquisition,
  automotive - just to name a few major use cases.

  RT development is backed by a Linuxfoundation project which is
  supported by major stakeholders of this technology. The funding will
  continue over the actual inclusion into mainline to make sure that the
  functionality is neither introducing regressions, regressing itself,
  nor becomes subject to bitrot. There is also a lifely user community
  around RT as well, so contrary to the grim situation 5 years ago, it's
  a healthy project.

  As RT is still a good vehicle to exercise rarely used code paths and
  to detect hard to trigger issues, you could at least view it as a QA
  tool if nothing else"
\end{verbatim}


\end{document}

